%%% A template to produce a nice-looking Curriculum Vitae.
%%% Kieran Healy <kjhealy@gmail.com>
%%% Most recent version is at http://kjhealy.github.com/kjh-vita
%%%
%%% ------------------------------------------------------------------------
%%% Requirements (should be included in a modern tex distribution):
%%% ------------------------------------------------------------------------
%%% xelatex
%%% fontspec.sty
%%% hyperrref.sty
%%% xunicode.sty
%%% color.sty
%%% url.sty
%%% fancyhdr.sty
%%%
%%% ------------------------------------------------------------------------
%%% Optional
%%% ------------------------------------------------------------------------
%%% git
%%% vc.sty
%%% revnum.sty
%%% Fonts
%%%
%%% ------------------------------------------------------------------------
%%% Note
%%%------------------------------------------------------------------------
%%% Because this is a hand-tweaked file, be on the look out for \medskip, 
%%% \bigskip and \newpage commands here and there, which are used to balance
%%% the layout or avoid widows & orphans, etc. You should of course add or 
%%% remove these as needed.
%%%------------------------------------------------------------------------

\documentclass[11pt,a4paper]{article}

\usepackage{cv}

%%%------------------------------------------------------------------------
%%% Metadata
%%%------------------------------------------------------------------------

%% Change as needed. Or just add me as a coauthor. Only some of these are 
%% used below in the hyperref declaration and address banner section.
\def\myauthor{James Keirstead}
\def\mytitle{Vita}
\def\mycopyright{\myauthor}
\def\mykeywords{}
\def\mybibliostyle{plain}
\def\mybibliocommand{}
\def\mysubtitle{}
\def\myaffiliation{Imperial College London}
\def\myaddress{Department of Civil and Environmental Engineering}
\def\myemail{j.keirstead@imperial.ac.uk}
\def\myweb{http://www.jameskeirstead.ca}
\def\myphone{+44 (0) 207 594 6010}
\def\myfax{+44 (0) 207 594 5934}
\def\myversion{}
\def\myrevision{}

\date{} % not used (revision control instead)
\def\mykeywords{James, Keirstead, James Keirstead, Vita, CV, Resume, Engineering}

%%%------------------------------------------------------------------------  
%%% Git version tracking 
%%%------------------------------------------------------------------------

%% If you don't use git or the vc package (from CTAN), comment this out.
%% If you comment it out, be sure to remove the \rfoot comment below, too.
%% See vc manual to compile with xelatex -enable-write18 vita
\immediate\write18{vc.bat}
\input{vc}

%%%------------------------------------------------------------------------
%%% Required style files
%%%------------------------------------------------------------------------
%%\usepackage{revnum} % for reverse-numbered publications (revnumerate environment) if needed.

%% needed for xelatex to work
\usepackage{fontspec}
\usepackage{xunicode}

%% color for the links 
\usepackage[usenames,dvipsnames]{xcolor}
\definecolor{ImperialBlue}{rgb}{0.082,0.416,0.608}
\definecolor{ImperialLightBlue}{rgb}{0.545,0.682,0.8}

% Bibliography stuff
\bibliography{../content/papers}
\addtocategory{books}{Keirstead2013a}
\addtocategory{papers}{Chester2014, Keirstead2014, Keirstead2014a, Keirsteada, Morlet_Keirstead_2013, Keirstead2011b, Rutter2012, Calderon2012, Keirstead2012b, Keirstead, Keirstead2011c, Keirstead2012a, Liang2012, Keirstead2010c, Liang2011, Keirstead2010b, Keirstead2009h, Keirstead2009b, Keirstead2008a, Keirstead2008, Keirstead2007b, Keirstead2007c, Keirstead2006a, Pfenninger2014a, Pfenninger2014, Pfenninger2013, Koppelaar2014}
\addtocategory{chapters}{Shah2012, Keirstead2013, Keirstead2012c, Grubler2011, Hammer2011, Schulzb, Keirstead2009f, Keirstead2010}
\addtocategory{conferences}{Giarola2013, Keirstead2011d, VanDam2011, VanDam2010, Weber2010, Keirstead2010d, Kostantinidis2010, Keirstead2010e, Keirstead2009g, Keirstead2009e, Keirstead2009c, Keirstead2009a, Keirstead2007, Keirstead2005b, Keirstead2005a, Baldock2015, Mijic2014, Sailer2015, Ravalde2014, Baedeker2014,  Pfenninger2014d, Walker2014, Pfenninger2014c, Pfenninger2014b, Keirstead2013b, Calderon2012a, Graham2012}
\addtocategory{other}{Keirstead2013c, Keirstead2013d, Keirstead2010a, Keirstead2008b, Keirstead2007a, Keirstead2005, Keirstead2010f, Keirstead2004, Keirstead2014b, Keirstead2014c, Keirstead2008c} 

%% hyperlinks
\hypersetup{xetex, 
	colorlinks=true,
	urlcolor=ImperialBlue,
	plainpages=false,
  	pdfpagelabels,
  	bookmarksnumbered,
  	pdftitle={\mytitle},
  	pagebackref,
  	pdfauthor={\myauthor},
  	pdfkeywords={\mykeywords}
  	}

%%%------------------------------------------------------------------------
%%% Document
%%%------------------------------------------------------------------------
\begin{document}

%% Choose fonts for use with xelatex
%% Minion and Myriad are widely available, from Adobe. 
%% Pragmata is available to buy at http://www.fsd.it/fonts/pragma.htm
%% and is worth every penny. Any good monospace font will work fine, though.
%% Consolas or inconsolata are good alternatives.
\setromanfont[Mapping={tex-text},Numbers={OldStyle},Ligatures={Common}]{Minion Pro} 
\setsansfont[Mapping=tex-text,Colour=156a9b]{Myriad Pro}
\setmonofont[Mapping=tex-text,Scale=0.9]{Avenir Next Condensed Regular} % Consolas, Pragmata, Lucida Console


%%%------------------------------------------------------------------------
%%% Local commands
%%%------------------------------------------------------------------------

%% Marginal header
%% Note: as the document goes on you may need to introduce a (gradually increasing)
%% \vspace element to keep the marginal header pleasingly aligned with the first 
%% item in the body text. Like this: \marginhead{{\vskip 0.4em}Grants}, or 
%% \marginhead{{\vskip 0.8em}Service}. Experiment as needed.
\newcommand{\marginhead}[1]{\marginpar{\textsf{{\footnotesize #1}}}}


%% custom ampersand (font consistent with the one chosen above)
\newcommand{\amper}{{\fontspec[Scale=.95,Colour=AA0000]{Minion Pro Medium}\selectfont\&\,}}

%% No bullets on labels
\renewcommand{\labelitemi}{~} 

%% Custom hanging indent for vita items
\def\ind{\hangindent=1 true cm\hangafter=1 \noindent}
%\def\ind{\hangindent=18pt\hangafter=1 \noindent}
\def\labelitemi{~}
\renewcommand{\labelitemii}{~}

%%%------------------------------------------------------------------------
%%% Page layout
%%%------------------------------------------------------------------------
\pagestyle{fancy}
\renewcommand{\headrulewidth}{0pt}
\fancyhead{}
\fancyfoot{}
\rhead{{\scriptsize\thepage}}

%% git revision control footer 
\rfoot{\textcolor{Gray}{\texttt{\scriptsize \VCRevision\ on \VCDateTEX}}} % git revision info inserted via external script -- see docs for vc package for details. comment out this line if you're not using vc, and also remove the \input{vc} line above.

%%%------------------------------------------------------------------------
%%% Address and contact block
%%%------------------------------------------------------------------------
\begin{minipage}[t]{2.95in}
\flushright {\footnotesize \href{http://www.imperial.ac.uk/civilengineering}{Department of Civil \& Env.\ Engineering} \\ Imperial College London \\ London, UK\\ \vspace{-0.04in} SW7 2AZ}   
\end{minipage}
\hfill     
%\begin{minipage}[t]{0.0in}
% dummy (needed here)
%\end{minipage}
\hfill
\begin{minipage}[t]{1.7in}
  \flushright \footnotesize Phone: \myphone \\ 
  Fax: \myfax  \\ 
  {\scriptsize  \texttt{\href{mailto:\myemail}{\myemail}}} \\
  {\scriptsize  \texttt{\href{\myweb}{\myweb}}}
\end{minipage}


\bigskip

\bigskip

%% Name 
\noindent{\Large {D\textsc{r} J\textsc{ames} K\textsc{eirstead}}} {\scriptsize \textsc{CEng MEI}}
\reversemarginpar
\raggedright

\bigskip

\bigskip

%% Appointments

\noindent\marginhead{Appointments}%
%
\emph{Imperial College London \vspace{0.01in}}

\ind 2011--Present. Lecturer, Dept.\ of Civil and Environmental Engineering.      

\ind 2009--2011. Research Fellow and Team Leader, BP Urban Energy Systems project.

\ind 2006--2009. Research Associate, BP Urban Energy Systems project. \vspace{0.02in}

\medskip

\emph{Institute of Industrial Science, University of Tokyo \vspace{0.01in}}

\ind October 2010.  Visiting Research Fellow.

\bigskip

%% Education

\noindent\marginhead{Education}%
%
\emph{University of Oxford \vspace{0.01in}}

\ind 2006.  DPhil, Energy Policy. \emph{Behavioural responses to photovoltaic systems in the UK domestic sector.}

\ind 2002.  MSc, Environmental Change and Management (Distinction).

\medskip

\noindent\emph{Queen's University, Kingston, Ontario, Canada\vspace{0.02in}}

\ind 2001. BSc Applied Science, Civil Engineering. (First Class Honours.) 

\bigskip
 
%% Publications
\begin{publications}
\printbib{books}
\printbib{papers}
% \ind Aruna Sivakumar, James Keirstead, JW Polak. 2012. ``Integrated Modelling of the Demand \& Supply Vectors in Urban Energy Systems: Conceptual and Modelling Frameworks for the Development of a New Toolkit'', Computer-Aided Civil and Infrastructure Engineering. http://onlinelibrary.wiley.com/journal/10.1111/%28ISSN%291467-8667
%
% \ind Nouri Samsatli, James Keirstead, and Nilay Shah. In review. ``A Generic MILP Model for the Design of Urban Energy Systems'' submitted to \emph{Applied Energy}.
%
%\ind James Keirstead and Mark Jennings. In review. ``Calculating carbon abatement costs for the global buildings sector with a regionally-disaggregated hybrid model'' submitted to \emph{Energy Economics}.  
%
% \ind James Keirstead and Niels Schulz.  Drafting. ``Scaling laws in urban energy consumption''.
%
% \ind I.\ Horta and J.\ Keirstead. 2015.  ``Downscaling'' JIE.
%
% \ind T.\ Ravalde and J.\ Keirstead. 2015. ``Metrics of metabolism.'' JCP.
%
% \ind C.\ Kennedy, I.\ Stewart, \emph{et al}. 2014. ``Energy and Material Flows of Megacities'' \emph{PNAS}, submitted.
%  
\printbib{chapters}
%\ind Various chapters in UES book?
\printbib{conferences}
%
% \ind T.\ Ravalde, J.\ Keirstead.  2015. ``Measuring the efficiency of urban metabolism'' ISIE oral presentation.  
%
% \ind T.\ Ravalde, J.\ Keirstead.  2015. ``Forecasting urban metabolism: scanning the horizon of urban processes.'' ISIE conference poster.  
%
% \ind S.\ Pfenninger, P.\ Gauché, J.\ Keirstead. 2015.  ``Comparing the whole-system costs of baseload power from CSP and nuclear power plants under high renewables scenarios for the case of South Africa''. IAEE 2015.
%
\printbib{other}
\end{publications}

%% Presentations
\noindent\marginhead{Invited Talks}%
%
\ind 2015. ``What is an urban energy system?''.  CASA Seminar Series, University College London.  London, February.

\ind 2014. ``What is an urban energy system?''.  Imperial College Energy Society seminar.  London, December.

\ind 2014. ``Benchmarking urban energy efficiency in the UK''.  Benchmarking Energy Sustainability in Cities JRC-EC Scientific Workshop, Turin, November.

\ind 2014. ``Smart Urban Energy Systems: From Grand Visions to Everyday Reality'' Cities, Energy, and Climate Change Mitigation conference, Leeds, July.

\ind 2014. ``Metabolic Impacts of the London Congestion Charge'' Gordon Research Conference on Industrial Ecology, June.

\ind 2014. ``A review of urban energy systems models: what role for weather and climate data?''. University of Reading,  Meteorology Dept, April.

\ind 2013. ``How do models of urban energy systems account for climate change?''. RCN Virtual Collaboratory Call, October.

\ind 2013. ``Technologies and policies for urban energy systems''.  Global Sustainability Summer School, Potsdam Institute for Climate Impact Research, Potsdam, Germany. July.

\ind 2013. ``Fit for purpose?  A comparison of urban resource demand simulation techniques''.  ISIE Conference, Ulsan, South Korea.

\ind 2013. ``What can dynamical systems tell us about urban energy systems?''.  SIAM Conference on the Application of Dynamical Systems, Snowbird, Utah. 

\ind 2012. ``What next for urban energy systems?'' Flexible Energy Delivery Systems seminar series, Cardiff University. October.

\ind 2012. ``Modelling urban energy systems with SynCity'' RCN Workshop on Sustainable Cities, Denver, USA.  August.

\ind 2012. ``Modelling urban energy systems'' GreenBridge seminar series, University of Cambridge, UK. March.

\ind 2012. ``Modelling approaches to urban energy systems: optimization, simulation, and more'' Martin Centre seminar series, University of Cambridge, UK. February.

\ind 2011. ``Modelling urban energy systems'' National Academy of Engineering/EU Frontiers of Engineering symposium, Irvine, CA. November.

\ind 2011. ``Modelling urban energy transitions'' Grantham Institute, Imperial College, London. September.

\ind 2011. ``Approaches, challenges and opportunities in urban energy systems modelling'' IIASA, Laxenburg, Austria. March.

\ind 2010. ``Challenges in multi-scale urban energy modelling and data collection: What's `just right' for urban energy data?'' \textsc{iis} seminar, University of Tokyo. November.

\ind 2010. ``Modelling energy systems with ontologies'' \textsc{iam} seminar, University of Southampton. June.

\ind 2010. ``SynCity: New Tools, Methods and Business Models for future urban sustainability'' Communities for Advanced Distributed Energy Resources, San Diego.  April.

\ind 2010. ``Lessons from SynCity: insights on the modelling of urban energy systems'' Gordon Conference on Industrial Ecology, New London, NH.  July.

\ind 2009. ``The SynCity layout model: exploring the limits of low-carbon urban design'' University of Tokyo, Tokyo.  September.

\ind 2009. ``Modelling urban energy systems with SynCity'' Nagoya University, Nagoya. February.

\ind 2008. ``SynCity: an integrated framework for modelling urban energy systems'' Asian Institute of Technology, Bangkok. February.

\ind 2007. ``Urban Energy Transitions'' National Institute for Environmental Studies, Tsukuba.  March.

%\end{revnumerate}

\bigskip

\noindent\marginhead{Grants and \newline Awards}%
%\medskip
%
% Two SACA Award nominations for Best Teaching for Taught Postgraduates.  Shortlist announced end of February.

%\ind 2014--present.  Modelling City Systems, Climate KIC.  Co-investigator (£?)
\ind 2014--present.  Future FM, EPSRC.  Co-investigator (£600,000).

\ind 2014--present.  Smart Sustainable District, Climate KIC.  Co-investigator (£68,000)

\ind 2014--present.  Sponsorship Award: Urban-scale Building Energy Network, EPSRC.  Principal investigator (£23,300).

\ind 2014--present. Downscaling urban metabolism, Enel Foundation/University of Toronto. Principal investigator (£10,000).

\ind 2012--present.  SusLabs NWE, EU Interregio.  Co-investigator (£170,000).

\ind 2012. Smart Thermal Storage Systems, DECC.  Co-investigator (£30,000).

\ind 2012. The cost of 2$^\circ$C, AREVA.  Co-investigator.

\ind 2012. Energy Futures Lab Research Challenge, Imperial College.  Principal investigator (£8,000).

% Commented out as I didn't do that much on it
%\ind 2011. Biomass system value-chain modelling, Energy Technologies Institute.  Co-investigator. (£153,000).
\ind 2011. Energy systems analysis of South Heaton, Newcastle University.  Co-investigator (£2,500).

\ind 2009. MBA Innovation, Entrepreneurship and Design, Imperial College.  Ideator (30 person-months research time).

\ind 2003--2006.  Commonwealth Scholar, held at University of Oxford.

\ind 2001--2002.  British Council Chevening Scholar, held at University of Oxford.

\ind 2001.  Consulting Engineers of Ontario Louanne Smrke Award for excellence in written communication.

\ind 2001. CW Marshall Award for excellence in structural engineering, Queen's University.

\ind 2000. Guinness Book of World Records, longest distance travelled in a solar vehicle (7044 km).

\ind 2000. One of 125 ``Canadians Shaping the Nation We Will Live in Tomorrow'', chosen by \emph{The Globe and Mail}.

\ind 2000. Fifth Field Company Prize for excellence in hydrology, Queen's University.

\ind 2000. Frederick and Christopher Ansley Award for extra-curriculars and academic excellence, Queen's University.

\ind 1999. \textsc{sn} Graham Award for extra-curriculars and academic excellence, Queen's University.

\ind 1998--2001.  Dean's Scholar for academic excellence, Queen's University.

\bigskip 

\noindent\marginhead{Service to the \newline Profession}%
\ind 2014. Panel member, Guardian Sustainable Business live chat on \href{http://www.theguardian.com/sustainable-business/energy-efficiency-commercial-properties-live-chat}{energy efficiency in non-domestic buildings}.

\ind 2014. Steering group member, Energy Research Partnership Cities project.

% \ind 2014. Associate editor, \emph{Infrastructure Complexity}.

\ind 2014. Programme committee member, First International Workshop on Demand Response (DR2014).

\ind 2013. Invited participant.  EPSRC Engineering Leaders of the Future. October, Hellidon Lakes.

\ind 2013. Invited participant.  Research Councils UK energy strategy workshop on energy infrastructure.  Birmingham.

\ind 2013. Technical committee member.  International Society for Industrial Ecology conference.

\ind 2011--present. Board member, Sustainable Urban Systems section, International Society for Industrial Ecology.

\ind 2011--2012. Technical advisory panel member, Challenging Lock-in Through Urban Energy Systems EPRSC project.

\ind 2011--2012. Technical committee member, CIBSE ASHRAE Technical Symposium 2012.

\ind 2011--2013. Programme committee member, 2nd--4th International Workshops on Agent Technologies for Energy Systems (ATES).

\ind 2010. Reviewer, ECOS and Next Generation Infrastructures conferences.

\ind 2009. Invited participant.  EPSRC Grand Challenge workshop on Sustainable Urban Environments.

\ind Reviewer, Engineering and Physical Sciences Research Council, Economic and Social Research Council, US National Science Foundation, Natural Sciences and Engineering Research Council of Canada, Social Sciences and Humanities Research Council of Canada, Paul Scherrer Institute .

\ind Reviewer, \emph{Energy Policy}, \emph{Proceedings of the National Academy of Sciences}, \emph{Building Research and Information}, \emph{Technological Forecasting and Social Change}, \emph{Energy}, \emph{Energy Economics}, \emph{Environmental Practice}, \emph{Journal of Artificial Societies and Social Simulation}, \emph{Energy Efficiency}, \emph{Sustainable: Science, Practice and Policy}, \emph{The Energy Journal}, \emph{Journal of Industrial Ecology}, \emph{Computers, Environment and Urban Systems}, \emph{Journal of Urban Technology}, \emph{International Journal of Sustainable Transportation}, \emph{Climatic Change}, \emph{Urban Design and Planning}, \emph{Environmental Science \& Technology}, \emph{IET Renewable Power Generation}, \emph{Environment and Planning B}, \emph{Ecological Modelling}, \emph{Environmental Research Letters}, \emph{Journal of Environmental Planning and Management}, \emph{Cities}, \emph{Sustainability Science}, \emph{Urban Studies}, \emph{Journal of Professional Issues in Engineering Education and Practice}, \emph{Solutions}, \emph{Civil Engineering and Environmental Systems}, \emph{Building and Environment}, \emph{Sustainable Cities and Society}.

\ind Consultant, Environmental Change Institute, University of Oxford (Carbon Trust, Department for International Development), Foresight (Sustainable Energy Management in the Built Environment, and Future of Cities), International Institute for Applied Systems Analysis, EDF/Weber Shandwick, The Economist Group, Energy Technologies Institute, Arup, University of Toronto/Enel Foundation.
\bigskip%

\newpage
\noindent\marginhead{Teaching}%
%
\ind 2013--present.  CI{\addfontfeatures{Numbers={Lining}}1-101} Drawing

\ind 2011--present.  CI{\addfontfeatures{Numbers={Lining}}1-182} Energy Systems

\ind 2011--present.  CI{\addfontfeatures{Numbers={Lining}}9-SD} Sustainable Development MSc module

\ind 2007--present. Various short lectures and modules on urban energy systems at both undergraduate and graduate levels.
 
\ind 2007--present. Supervised or co-supervised 29 MSc projects, 5 MEng projects and internships.

\bigskip
\noindent\marginhead{PhD students}%
%%
% Simon de Sterke
\ind 2014--present.  Peter North. \emph{Heat energy strategy for London}.

\ind 2014--present.  Honor Brixby. \emph{Urbanisation and non-communicable diseases}.  Joint with Majid Ezzati.

\ind 2014--present.  Sarah Noy\'{e}. \emph{Wireless sensor networks for post-occupancy building commissioning}.  

\ind 2012--present.  Stefan Pfenninger. \emph{Multi-scale energy systems}

\ind 2012--present.  Tom Ravalde. \emph{Highly integrated urban resource systems}

\ind 2012--present.  Rembrandt Koppelaar. \emph{Expert knowledge development of property allocation and technological change}.

\ind 2012--present.  Pantelis Broukos. \emph{Multi-period urban energy systems optimization in the context of water and wastewater management in Luxembourg}
\bigskip

%\newpage
\noindent\marginhead{Affiliations}%
%
\ind 2010--present.  Member and Chartered Energy Engineer, Energy Institute.

\ind 2010--present.  Member, International Society for Industrial Ecology.

\ind 2009--present.  Associate, Higher Education Academy.

\ind 2008--present.  Member, British and International Institutes for Energy Economics.

\end{document}
