%%% A template to produce a nice-looking Curriculum Vitae.
%%% Kieran Healy <kjhealy@gmail.com>
%%% Most recent version is at http://kjhealy.github.com/kjh-vita
%%%
%%% ------------------------------------------------------------------------
%%% Requirements (should be included in a modern tex distribution):
%%% ------------------------------------------------------------------------
%%% xelatex
%%% fontspec.sty
%%% hyperrref.sty
%%% xunicode.sty
%%% color.sty
%%% url.sty
%%% fancyhdr.sty
%%%
%%% ------------------------------------------------------------------------
%%% Optional
%%% ------------------------------------------------------------------------
%%% git
%%% vc.sty
%%% revnum.sty
%%% Fonts
%%%
%%% ------------------------------------------------------------------------
%%% Note
%%%------------------------------------------------------------------------
%%% Because this is a hand-tweaked file, be on the look out for \medskip, 
%%% \bigskip and \newpage commands here and there, which are used to balance
%%% the layout or avoid widows & orphans, etc. You should of course add or 
%%% remove these as needed.
%%%------------------------------------------------------------------------

% This is a much nicer way of handling all the references, etc.
% as it will just pull them in from a BibTeX file, which of course I
% can link to Mendeley
% 
% http://robjhyndman.com/hyndsight/cv/

\documentclass[11pt,a4paper]{article}

%%%------------------------------------------------------------------------
%%% Metadata
%%%------------------------------------------------------------------------

%% Change as needed. Or just add me as a coauthor. Only some of these are 
%% used below in the hyperref declaration and address banner section.
\def\myauthor{James Keirstead}
\def\mytitle{Vita}
\def\mycopyright{\myauthor}
\def\mykeywords{}
\def\mybibliostyle{plain}
\def\mybibliocommand{}
\def\mysubtitle{}
\def\myaffiliation{Imperial College London}
\def\myaddress{Department of Civil and Environmental Engineering}
\def\myemail{j.keirstead@imperial.ac.uk}
\def\myweb{http://www.jameskeirstead.ca}
\def\myphone{+44 (0) 207 594 6010}
\def\myfax{+44 (0) 207 594 5934}
\def\myversion{}
\def\myrevision{}

\date{} % not used (revision control instead)
\def\mykeywords{James, Keirstead, James Keirstead, Vita, CV, Resume, Engineering}

%%%------------------------------------------------------------------------  
%%% Git version tracking 
%%%------------------------------------------------------------------------

%% If you don't use git or the vc package (from CTAN), comment this out.
%% If you comment it out, be sure to remove the \rfoot comment below, too.
%% See vc manual to compile with xelatex -enable-write18 vita
\immediate\write18{vc.bat}
\input{vc}

%%%------------------------------------------------------------------------
%%% Required style files
%%%------------------------------------------------------------------------
\usepackage{url,fancyhdr}
\usepackage{etaremune} % for reverse-numbered publications (revnumerate environment) if needed.

%% needed for xelatex to work
\usepackage{fontspec}
\usepackage{xunicode}

%% color for the links 
\usepackage[usenames,dvipsnames]{xcolor}
\definecolor{ImperialBlue}{rgb}{0.082,0.416,0.608}
\definecolor{ImperialLightBlue}{rgb}{0.545,0.682,0.8}

%% hyperlinks
\usepackage[xetex, 
	colorlinks=true,
	urlcolor=ImperialBlue,
        linkcolor=ImperialBlue,
	plainpages=false,
  	pdfpagelabels,
  	bookmarksnumbered,
  	pdftitle={\mytitle},
  	pagebackref,
  	pdfauthor={\myauthor},
  	pdfkeywords={\mykeywords}
  	]{hyperref}

%% Choose fonts for use with xelatex
%% Minion and Myriad are widely available, from Adobe. 
%% Pragmata is available to buy at http://www.fsd.it/fonts/pragma.htm
%% and is worth every penny. Any good monospace font will work fine, though.
%% Consolas or inconsolata are good alternatives.
\setromanfont[Mapping={tex-text},Numbers={OldStyle},Ligatures={Common}]{Minion Pro} 
\setsansfont[Mapping=tex-text,Colour=156a9b]{Myriad Pro}
\setmonofont[Mapping=tex-text,Scale=0.9]{Consolas} % Pragmata, Lucida Console


%%%------------------------------------------------------------------------
%%% Local commands
%%%------------------------------------------------------------------------

%% Marginal header
%% Note: as the document goes on you may need to introduce a (gradually increasing)
%% \vspace element to keep the marginal header pleasingly aligned with the first 
%% item in the body text. Like this: \marginhead{{\vskip 0.4em}Grants}, or 
%% \marginhead{{\vskip 0.8em}Service}. Experiment as needed.
\newcommand{\marginhead}[1]{\marginpar{\textsf{{\footnotesize #1}}}}


%% custom ampersand (font consistent with the one chosen above)
\newcommand{\amper}{{\fontspec[Scale=.95,Colour=AA0000]{Minion Pro Medium}\selectfont\&\,}}

%% No bullets on labels
\renewcommand{\labelitemi}{~} 

\newcommand{\flag}{\marginpar{\hfill\raisebox{-2pt}{*}}}
%% Custom hanging indent for vita items
\def\ind{\hangindent=1 true cm\hangafter=1 \noindent}
%\def\ind{\hangindent=18pt\hangafter=1 \noindent}
\def\labelitemi{~}
\renewcommand{\labelitemii}{~}

%%%------------------------------------------------------------------------
%%% Page layout
%%%------------------------------------------------------------------------
\pagestyle{fancy}
\renewcommand{\headrulewidth}{0pt}
\fancyhead{}
\fancyfoot{}
\rhead{{\scriptsize\thepage}}
%% git revision control footer 
\rfoot{\textcolor{Gray}{\texttt{\scriptsize \VCRevision\ on \VCDateTEX}}} % git revision info inserted via external script -- see docs for vc package for details. comment out this line if you're not using vc, and also remove the \input{vc} line above.

\fancypagestyle{plain}%
%\renewcommand{\headrulewidth}{0pt}
\fancyhead{}
\fancyfoot{}
%%% git revision control footer 
\rfoot{\textcolor{Gray}{\texttt{\scriptsize \VCRevision\ on \VCDateTEX}}} % git revision info inserted via external script -- see docs for vc package for details. comment out this line if you're not using vc, and also remove the \input{vc} line above.


%%%------------------------------------------------------------------------
%%% Document
%%%------------------------------------------------------------------------
\begin{document}


\thispagestyle{plain}

%%%------------------------------------------------------------------------
%%% Address and contact block
%%%------------------------------------------------------------------------
\begin{minipage}[t]{2.95in}
\flushright {\footnotesize \href{http://www.imperial.ac.uk/civilengineering}{Department of Civil \& Env.\ Engineering} \\ Imperial College London \\ London, UK\\ \vspace{-0.04in} SW7 2AZ}   
\end{minipage}
\hfill     
%\begin{minipage}[t]{0.0in}
% dummy (needed here)
%\end{minipage}
\hfill
\begin{minipage}[t]{1.7in}
  \flushright \footnotesize Phone: \myphone \\ 
  Fax: \myfax  \\ 
  {\scriptsize  \texttt{\href{mailto:\myemail}{\myemail}}} \\
  {\scriptsize  \texttt{\href{\myweb}{\myweb}}}
\end{minipage}


\bigskip

\bigskip

%% Name 
\noindent{\Large {D\textsc{r} J\textsc{ames} K\textsc{eirstead}}} {\scriptsize \textsc{CEng MEI}}
\reversemarginpar

\bigskip

\bigskip


\noindent{\Large \emph{Publications List}}

\bigskip

\noindent\marginhead{Statement of\\Significance}%
My research has been in three fields.  First, I examined energy consumption behaviour in the domestic sector  with a particular focus on microgeneration technologies such as small-scale solar photovoltaics.  Paper \ref{paper:dphil} was the main output from this work and is my most highly cited paper to date (93 citations).  Second, I have worked on governance issues associated with urban energy consumption.  This has lead to methodological developments in measuring urban performance as well as comparative studies of urban energy governance in Europe (e.g.\ Paper~\ref{paper:benchmarking}).  Finally my most recent work has developed integrated modelling approaches for urban energy systems, with an emphasis on optimisation techniques.  This work has been featured in several international assessments (see Book Contributions \ref{book:uccrn}--\ref{book:urban_opt}), journal articles, and a major book (\ref{book:ues}). 

\begin{flushright}
\itshape h-index: 11
\medskip

All citation figures refer to values from Google Scholar as of 13 August 2014.  Significant works are marked in the margin with \raisebox{-2pt}{*}.
\end{flushright}

\bigskip

\raggedright 
%% Publications
\noindent\marginhead{Books}%
%
%\medskip
\noindent\emph{Books Authored and Edited \vspace{0.01in}}

\begin{enumerate}
\item Keirstead, J.\ \flag\ and Shah, N.\ (editors).  2013. \emph{Urban Energy Systems: an Integrated Approach}. London: Routledge/Earthscan.\label{book:ues}
%\normalsize
\end{enumerate}

\bigskip

\noindent\emph{Contibutions to Books Edited by Others \vspace{0.01in}}
% \renewcommand{\labelenumi}{\textsc{c}\theenumi.}
\begin{enumerate}
\item Keirstead, J. 2010. ``Total primary energy supply'' in \emph{Green Energy: An A-to-Z Guide}, edited by P.\ Robbins and D.\ Mulvaney.  Berkeley, CA: SAGE Publications, 426--428.

\item Keirstead, J., Samsatli, N.\  and Shah, N. 2010. ``SynCity: an integrated tool kit for urban energy systems modelling'' in \emph{Energy Efficient Cities: Assessment Tools and Benchmarking Practices}, edited by R.\ Bose.  Washington: World Bank, 21--42.

\item Schulz, N., Shah, N., Fisk, D., Keirstead, J., Samsatli, N., Sivakumar, A., Weber, C., and Saunders, E. 2010. ``Mobilize: The SynCity Urban Energy Systems Model'' in \emph{Ecological Urbanism}, edited by M.\ Mostafavi and G.\ Doherty.  Baden, Lars M\"{u}ller Publishers.

\item Hammer, S., Keirstead, J., Dhakal, S., Mitchell, J., Colley, M., Gonzalez, R., Herv\'{e}-Mignucci, M., Parshall, L., Schulz, N., Hyams, M. 2011. ``Climate Change and Urban Energy Systems'' in \emph{Climate Change and Cities: First Assessment Report of the Urban Climate Change Research Network}, edited by C.\ Rosenzweig, W.D.\ Solecki, S.A.\ Hammer, S.\ Mehrotra.  Cambridge: Cambridge University Press, 85--112.\label{book:uccrn}

\item Grubler, A., Bai, X., Buettner, T., Dhakal, S., Fisk, D., Ichinose, T., Keirstead, J., Sammer, G., Satterthwaite, D., Schulz, N., Shah, N., Steinberger, J., and Weisz, H. 2012. ``Urban Energy Systems'' in \emph{The Global Energy Assessment}. Cambridge: Cambridge University Press.

\item Keirstead, J.\  and Shah, N. 2012. ``Urban Energy Systems Planning, Design and Implementation'' in \emph{Energizing Sustainable Cities: Assessing Urban Energy}, edited by A.\ Grubler and D.\ Fisk. London: Earthscan/Routledge.

\item Keirstead, J.\  and Shah, N. 2013. ``The changing role of optimization in urban planning'' in \emph{Optimization, Simulation and Control}, edited by A.\ Chinchuluun, P.M.\ Pardalos, R.\ Enkhbat, S.\ Pistikopoulos.  Springer Series in Optimization and Its Applications.  Springer.\label{book:urban_opt}


%\item Various chapters in UES book?

\end{enumerate}

\bigskip 

\noindent\marginhead{Journal articles}%
%\emph{Journal articles \vspace{0.01in}} % 0.05
%
%% Use revnumerate environment if numbered publications are needed. 
%% (Include it above in the preamble).
%% \renewcommand{\labelenumi}{\textsc{a}\theenumi.}
%% \begin{revnumerate}
%  
\mbox{}\vspace{-2em}
\begin{enumerate}

\item Keirstead, J. 2006. ``\href{http://dx.doi.org/10.1016/j.enpol.2005.06.004}{Evaluating the applicability of integrated domestic energy frameworks in the UK}'' \emph{Energy Policy} 34(17):~3065--3077.

\item Keirstead, J. 2007. ``\href{http://dx.doi.org/10.1016/j.enpol.2006.08.003}{The UK domestic photovoltaics industry and the role of central government}'' \emph{Energy Policy} 35(4):~2268--2280.

\item Keirstead, J.\flag\ 2007. ``\href{http://dx.doi.org/10.1016/j.enpol.2007.02.019}{Behavioural responses to photovoltaic systems in the UK domestic sector}'' \emph{Energy Policy} 35(8):~4128--4141.\label{paper:dphil}

\item Keirstead, J. 2008. ``\href{http://dx.doi.org/10.1016/j.enpol.2008.09.019}{What changes, if any, would increased levels of low-carbon decentralised energy have on the built environment?}'' \emph{Energy Policy} 36(12):~4518--4521.

\item Keirstead, J.\  and Leach, M.  2008. ``\href{http://dx.doi.org/10.1002/sd.349}{Bridging the gaps between theory and practice: a service niche approach to urban sustainability indicators}'' \emph{Sustainable Development} 16(5):~329--340.

\item Keirstead, J. 2009. ``\href{http://dx.doi.org/10.1016/j.eiar.2008.09.001}{Feeling lucky? Using search engines to assess perceptions of urban sustainability}'' \emph{Environmental Impact Assessment Review} 29(2):~87--95.

\item Keirstead, J.\  and Schulz, N. 2010. ``\href{http://dx.doi.org/10.1016/j.enpol.2009.07.025}{London and beyond: Taking a closer look at urban energy policy}'' \emph{Energy Policy} 38(9):~4870--4879.

\item Keirstead, J. 2010. ``\href{http://dx.doi.org/10.1108/14777831011010829}{Applying service niche indicators to London's energy system}'' \emph{Management of Environmental Quality: An International Journal} 21(1):~6--19.

\item Keirstead, J.\  and Shah, N.\flag\ 2011. ``\href{http://dx.doi.org/10.1016/j.compenvurbsys.2010.12.005}{Calculating minimum energy urban layouts with mathematical programming and Monte Carlo analysis techniques}.'' \emph{Computers, Environment and Urban Systems} 35(5):~368--377. 

\item Liang, H., Keirstead, J., Samsatli, N., Shah, N., and Long, W. 2011. ``Application of a novel, optimisation-based toolkit (`SynCity') for urban energy system design in Shanghai Lingang New City.'' \emph{Energy Education Science and Technology Part A}, 28(1):~311--318.

\item Liang, H., Long, W., Keirstead, J., Samsatli, N., and Shah, N. 2012. ``\href{http://dx.doi.org/10.4028/www.scientific.net/AMR.433-440.1338}{Urban Energy System Planning and Chinese Low-Carbon Eco-City Case Study}''. \emph{Advanced Materials Research}, 433--440: 1338--1345. 

\item Keirstead, J., Samsatli, N., Pantaleo, A.M.\, and Shah, N. 2012. ``\href{http://dx.doi.org/10.1016/j.biombioe.2012.01.022}{Evaluating biomass energy strategies for a UK eco-town with an MILP optimization model}.'' \emph{Biomass and Bioenergy}, 39: 306--316.

\item Keirstead, J., Samsatli, N., Shah, N.\  and Weber, C. 2012. ``\href{http://dx.doi.org/10.1016/j.energy.2011.06.011}{The impact of CHP (combined heat and power) planning restrictions on the efficiency of urban energy systems}.'' \emph{Energy}, 41(1): 93--103.

\item Keirstead, J., Jennings, M., and Sivakumar, A. 2012. ``\href{http://dx.doi.org/10.1016/j.rser.2012.02.047}{A review of urban energy system models: approaches, challenges, and opportunities}.'' \emph{Renewable and Sustainable Energy Reviews}, 16(6): 3847--3866.

\item Keirstead, J.\  and Calderon, C.\flag\  2012. ``\href{http://dx.doi.org/10.1016/j.enpol.2012.03.058}{Capturing spatial effects, technology interactions, and uncertainty in urban energy and carbon models: retrofitting Newcastle as a case-study}.''  \emph{Energy Policy}, 46: 253--267.

\item Calderon, C.\  and Keirstead, J. 2012. ``\href{http://dx.doi.org/10.1080/09613218.2012.680702}{Modelling frameworks for delivering low carbon cities: advocating a normalised practice}''. \emph{Building Research and Information}, 40(4): 504--517.

\item Rutter, P.\ and Keirstead, J.  2012. ``\href{http://dx.doi.org/10.1016/j.enpol.2012.03.072}{A brief history and the possible future of urban energy systems}''. \emph{Energy Policy}, 50: 72--80.

\item Keirstead, J.\  and Sivakumar, A. 2012. ``\href{http://dx.doi.org/10.1111/j.1530-9290.2012.00486.x}{Using activity-based modeling to simulate urban resource demands at high spatial and temporal resolutions}''. \emph{Journal of Industrial Ecology}, 16(6): 889--900.

\item Morlet, C.\ and Keirstead, J. 2013. ``\href{http://dx.doi.org/10.1016/j.enpol.2013.06.085}{A comparative analysis of urban energy governance in four European cities}'' \emph{Energy Policy}, 61: 852--863.

\item Keirstead, J. 2013. ``\href{http://dx.doi.org/10.1016/j.enpol.2013.08.063}{Benchmarking urban energy efficiency in the UK}'' \emph{Energy Policy}, 63: 575--587.\label{paper:benchmarking}

\item Keirstead, J. 2014. ``\href{http://dx.doi.org/10.1111/jiec.12093}{Fit for purpose? Rethinking modeling in industrial ecology?}'' \emph{Journal of Industrial Ecology}. 18(2): 161--163.

\item Pfenninger, S., Hawkes, A., and Keirstead, J. 2014. ``\href{http://dx.doi.org/10.1016/j.rser.2014.02.003}{Energy systems modeling for twenty-first century energy policy}'' \emph{Renewable and Sustainable Energy Reviews}, 33: 74--86.

\item Keirstead, J.  2014.  ``\href{http://dx.doi.org/10.1680/ensu.13.00036}{Introducing sustainable development with a mathematical model}'' \emph{ICE Engineering Sustainability}, in press.

\item Chester, M., Sperling, J., Stokes, E., Allenby, B., Kochelman, K., Kennedy, C., Baker, L., Keirstead, J., Hendrickson, C. 2014. ``Positioning Infrastructure and Technologies for Low-carbon Urbanization.'' \emph{Earth's Future}, accepted.

% Buildings and Environment with Carlos
% DisP paper with Carlos
%
% \item Sivakumar, A., Keirstead, J., JW Polak. 2012. ``Integrated Modelling of the Demand \& Supply Vectors in Urban Energy Systems: Conceptual and Modelling Frameworks for the Development of a New Toolkit'', Computer-Aided Civil and Infrastructure Engineering. http://onlinelibrary.wiley.com/journal/10.1111/%28ISSN%291467-8667
%
% \item Samsatli, N., Keirstead, J., and Shah, N. In review. ``A Generic MILP Model for the Design of Urban Energy Systems'' submitted to \emph{Applied Energy}.
%
%\item Keirstead, J.\  and Jennings, M. In review. ``Calculating carbon abatement costs for the global buildings sector with a regionally-disaggregated hybrid model'' submitted to \emph{Energy Economics}.  
%
% \item Keirstead, J.\  and Niels Schulz.  Drafting. ``Scaling laws in urban energy consumption''.
%
% \item C.\ Kennedy, I.\ Stewart, \emph{et al}. 2014. ``Energy and Material Flows of Megacities'' \emph{Nature}, submitted.
% 
% \item S.\ Pfenninger and Keirstead, J. 2014 ``Renewables, nuclear, or fossil fuels? Comparing scenarios for the UK electricity system''.  \emph{Applied Energy}.  To be submitted shortly.
%
% \item R.\ Koppelaar, Keirstead, J., J.\ Woods, N.\ Shah. 2014. ``A review of policy analysis purposes and capabilities of electricity system models'' \emph{Renewable and Sustainable Energy Reviews}, submitted.

\end{enumerate}

\bigskip


\noindent\marginhead{Conference papers}%
% 
\emph{Peer-reviewed proceedings only}
% AGU 2014, Ana's abstract?
%
% IST Conference 5th International Conference on Sustainability Transitions, http://www.uu.nl/faculty/geosciences/EN/IST2014/Pages/default.aspx, led by Najine Ameli?
%
%\item T.\ Ravalde and Keirstead, J. 2014. ``Title''. International Symposium for Next Generation Infrastructure, Vienna, Austria, November.

\begin{enumerate}

\item Keirstead, J. 2005. ``Photovoltaics in the UK domestic sector: a double-dividend?'' in \emph{Proceedings of the ECEEE Summer Study}, edited by F.\ Bartiaux and A.\ Saln{\ae}s. Mandelieu, France, 1249--1258.

\item Keirstead, J. 2005. ``Household behavioural responses to photovoltaic-system monitoring devices'' in \emph{Proceedings of the World Renewable Energy Congress}, edited by M.\ Imbabi and C.\ Mitchell. Aberdeen, 458--463.

\item Keirstead, J. 2007. ``Selecting sustainability indicators for urban energy systems'' in \emph{Proceedings of the International Conference on Whole Life Urban Sustainability and its Assessment}, edited by M.\ Horner, C.\ Hardcastle, A.\ Price, J.\ Bebbington. Glasgow, 1--20.

\item Keirstead, J. 2009. ``London's energy system: prospects for using the service niche approach in current indicator practice'' in \emph{Proceedings of the Second International Conference on Whole Life Urban Sustainability and its Assessment}, edited by M.\ Horner, A.\ Price, J.\ Bebbington, R.\ Emmanuel.  Loughborough, 488--502.

\item Keirstead, J. 2009. ``London's energy system: assessing the quality of urban sustainability indicators using the service niche approach'' in \emph{Proceedings of the Second International Conference on Whole Life Urban Sustainability and its Assessment}, edited by M.\ Horner, A.\ Price, J.\ Bebbington, R.\ Emmanuel.  Loughborough, 471--487.

\item Keirstead, J., Samsatli, N., Pantaleo, A.M., and Shah, N. 2009. ``Evaluating integrated urban biomass strategies for a UK eco-town'' in \emph{Proceedings of the European Biomass Conference}.  Hamburg, 2115--2127.

\item Keirstead, J., Samsatli, N.\  and Shah, N. 2009. ``SynCity: an integrated tool kit for urban energy systems modelling'' in \emph{Proceedings of the 5th Urban Research Symposium}.  Marseilles, 1--19.

\item Keirstead, J.\  and van Dam, K. 2010. ``A comparison of two ontologies for agent-based modelling of energy systems'' in \emph{Proceedings of the First International Workshop on Agent Technologies for Energy Systems (ATES 2010)}, edited by A.\ Rogers. Toronto, 1--8. 

\item Kostantindis, M., Samsatli, N., Keirstead, J.\  and Shah, N. 2010. ``Modelling of integrated municipal solid waste to energy technologies in the urban environment'' in \emph{Proceedings of the 3rd International Conference on Engineering for Waste and Biomass Valorisation}.  Beijing, 1--6.

\item Keirstead, J., Samsatli, N., Shah, N.\  and Weber, C. 2010. ``The implications of CHP planning restrictions on the efficiency of urban energy systems'' in \emph{Proceedings of the 23rd International Conference on Efficiency, Cost, Optimization, Simulation and Environmental Impact of Energy Systems}.  Lausanne, 1--8.

\item Weber, C., Keirstead, J., Samsatli, N., Shah, N.\  and Fisk, D. 2010. ``Trade-offs between Layout of Cities and Design of District Energy Systems'' in \emph{Proceedings of the 23rd International Conference on Efficiency, Cost, Optimization, Simulation and Environmental Impact of Energy Systems}.  Lausanne, 1--8.

\item van Dam, K.\ and Keirstead, J. 2010. ``Re-use of an ontology for urban energy systems modelling'' in \emph{Proceedings of Next Generation Infrastructures Eco-Cities conference}.  Shenzhen, 1--6.

\item Acha, S., van Dam, K., Keirstead, J.\  and Shah, N. 2011. ``Integrated modelling of agent-based electric vehicles into optimal power flow studies'' in \emph{Proceedings of the 21st International Conference and Exhibition on Electricity Distribution (CIRED 2011)}.  Frankfurt, 1--4.

\item Keirstead, J.\  and van Dam, K. 2011. ``A survey on the application of conceptualisations in energy systems modelling'' in \emph{Formal Ontologies Meet Industry: Proceedings of the Fifth International Workshop}, edited by P.E.\ Vermass and V.\ Dignum.  Amsterdam: IOS Press, 50--62.

\item Calderon, C.\  and Keirstead, J. 2012. ``Modelling approaches for retrofitting energy systems: Newcastle as a case study''.  Applied Urban Modelling 2012, Cambridge.

\item Giarola, S., Pantaleo, A.M., Keirstead, J., and Shah, N. 2013. ``Biomass and Natural Gas Cofiring Strategies for Optimal Integration into Urban Energy Systems''.  European Biomass Conference, Copenhagen.

\item Graham, N.J.D., Hellgardt, K., Wong, K.S., Keirstead, J., and Majali, Q. 2013. ``Combining H$_2$ Generation and Grey Water Recycling at Household Scale'' 12th International Conference on Sustainable Energy Technologies, Hong Kong.

\item Keirstead, J. 2013. ``Introducing sustainable development to engineers with a simple mathematical model''.  Engineering Education for Sustainable Development conference.  Cambridge.

\item Pfenninger, S.\ and Keirstead, J. 2014. ``Contrasting different electricity futures with Calliope, a modular, high-resolution, parallel and open-source energy systems model''.  IQ SCENE: Innovative Scenarios Techniques workshop, London.

\item Pfenninger, S.\ and Keirstead, J. 2014. ``Evaluating scenarios for the UK electricity system with a high-resolution energy systems model'' SET For Britain, London.

\item Walker, S.L.\ and Keirstead, J. 2014. ``Localised impacts of national low carbon energy scenarios.''  Energy Systems Conference, London.

\item Pfenninger, S.\ and Keirstead, J. 2014. ``Contrasting different UK electricity futures with a high-resolution energy systems model.''  Energy Systems Conference, London.

\item Baedeker, C., Greiff, K., Grinewitschus, V., Hasselku{\ss}, M., Keirstead, J., Keyson, D., Knutsson, J., Liedtke, J., Lockton, D., Lovric, T., Morrison, G., van Rijn, M., Rohn, H., Silvester, S., van Harinxma, W., Virdee, L. 2014. ``Transition through sustainable product and service innovations in sustainable living labs: application of user-centred research methodology within four living labs in Northern Europe''.  5th International Sustainable Transitions Conference, Utrecht, August. 

\item Ravalde, T., Keirstead, J. 20140. ``Integrated resource planning for a Chinese urban development''. International Symposium for Next Generation Infrastructure, Vienna, Austria, September.

\end{enumerate}

\bigskip

\noindent\marginhead{Reports and\\other writing}%
% 
\emph{Peer-reviewed only}
\medskip
%\newpage
\begin{enumerate}
\item Shah, N., Vallejo, L., Cockerill, T., Gambhir, A., Heyes, A., Hills, T., Jennings, M., Jones, O., Kalas, N., Keirstead, J., Khor, C., Mazur, C., Napp, T., Strapasson, A., Tong, D., and Woods, J. 2013. ``\href{http://www3.imperial.ac.uk/climatechange/publications/collaborative/halving-global-co2-by-2050}{Halving Global CO2 by 2050: Technologies and Costs}''.  Imperial College London.


\item Keirstead, J. 2010. ``Identifying lessons for energy-efficient cities using an integrated urban energy systems model''.  Working paper prepared for \emph{The Global Energy Assessment}.  Laxenburg, Austria: IIASA, 1--23.

\item Keirstead, J. 2010. ``\href{http://jasss.soc.surrey.ac.uk/13/1/reviews/keirstead.html}{Review} of Ball, Michael and Wietschel, Martin (eds): The Hydrogen Economy: Opportunities and Challenges'' \emph{Journal of Artificial Societies and Social Simulation}, 13(1).


\end{enumerate}

\noindent\marginhead{Software}%
% 
\emph{Peer-reviewed only}

%\newpage
\begin{enumerate}
\item Keirstead, J. 2013. ``scholar: Analyse citation data from Google Scholar''. R package version 0.1.1, \url{http://cran.r-project.org/web/packages/scholar/}.

\item Keirstead, J. 2014. ``decctools: tools for accessing UK energy statistics''. R package version 0.2.0, \url{http://cran.r-project.org/web/packages/decctools/}.
\end{enumerate}

%\noindent\emph{Reviews, software and other writing \vspace{0.05in}}

%\renewcommand{\labelenumi}{\textsc{r}\theenumi.}
%\begin{revnumerate}

% % \end{revnumerate}
 \bigskip

\noindent\marginhead{Non-peer\\reviewed}
%
% 
\emph{Software}
\medskip
\begin{enumerate}
\item Keirstead, J. 2014. ``jE++: a streamlined interface to jEPlus''. Java software, version 0.1.1, \url{https://github.com/keirstead-group/jeplusplus}.

\item Keirstead, J. 2014. ``SimElec: simulation model of domestic electricity demands''. Java software, version 0.1.2, \url{https://github.com/keirstead-group/simelec/}.
\end{enumerate}

\emph{Other writing}

\begin{enumerate}
\item Keirstead, J. 2004. ``Negotiating lifestyles: understanding lifestyles can help achieve sustainability''. \emph{IISD Alumni Inside}.

\item Keirstead, J. 2005. ``Energy matters: Power from the sun''.  \emph{Geography Review}, September 2005: 17--19.

\item Keirstead, J. 2007.  \href{http://www3.imperial.ac.uk/pls/portallive/docs/1/24897696.PDF}{\emph{Towards urban energy system indicators}}.  Internal report. London: Imperial College.

\item Keirstead, J. 2008. ``\href{http://www.stockholm-network.org/downloads/publications/Climate_of_Opinion_10.pdf}{Victims or leaders? Cities and global climate policy}''. \emph{Climate of Opinion}, 10:~4--5.

\end{enumerate}

\end{document}
