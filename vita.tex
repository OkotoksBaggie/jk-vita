%%% A template to produce a nice-looking Curriculum Vitae.
%%% Kieran Healy <kjhealy@gmail.com>
%%% Most recent version is at http://kjhealy.github.com/kjh-vita
%%%
%%% ------------------------------------------------------------------------
%%% Requirements (should be included in a modern tex distribution):
%%% ------------------------------------------------------------------------
%%% xelatex
%%% fontspec.sty
%%% hyperrref.sty
%%% xunicode.sty
%%% color.sty
%%% url.sty
%%% fancyhdr.sty
%%%
%%% ------------------------------------------------------------------------
%%% Optional
%%% ------------------------------------------------------------------------
%%% git
%%% vc.sty
%%% revnum.sty
%%% Fonts
%%%
%%% ------------------------------------------------------------------------
%%% Note
%%%------------------------------------------------------------------------
%%% Because this is a hand-tweaked file, be on the look out for \medskip, 
%%% \bigskip and \newpage commands here and there, which are used to balance
%%% the layout or avoid widows & orphans, etc. You should of course add or 
%%% remove these as needed.
%%%------------------------------------------------------------------------

\documentclass[11pt,a4paper]{article}

%%%------------------------------------------------------------------------
%%% Metadata
%%%------------------------------------------------------------------------

%% Change as needed. Or just add me as a coauthor. Only some of these are 
%% used below in the hyperref declaration and address banner section.
\def\myauthor{James Keirstead}
\def\mytitle{Vita}
\def\mycopyright{\myauthor}
\def\mykeywords{}
\def\mybibliostyle{plain}
\def\mybibliocommand{}
\def\mysubtitle{}
\def\myaffiliation{Imperial College London}
\def\myaddress{Department of Civil and Environmental Engineering}
\def\myemail{j.keirstead@imperial.ac.uk}
\def\myweb{http://www.jameskeirstead.ca}
\def\myphone{+44 (0) 207 594 6010}
\def\myfax{+44 (0) 207 594 5934}
\def\myversion{}
\def\myrevision{}

\date{} % not used (revision control instead)
\def\mykeywords{James, Keirstead, James Keirstead, Vita, CV, Resume, Engineering}

%%%------------------------------------------------------------------------  
%%% Git version tracking 
%%%------------------------------------------------------------------------

%% If you don't use git or the vc package (from CTAN), comment this out.
%% If you comment it out, be sure to remove the \rfoot comment below, too.
%% See vc manual to compile with xelatex -enable-write18 vita
\immediate\write18{vc.bat}
\input{vc}

%%%------------------------------------------------------------------------
%%% Required style files
%%%------------------------------------------------------------------------
\usepackage{url,fancyhdr}
%%\usepackage{revnum} % for reverse-numbered publications (revnumerate environment) if needed.

%% needed for xelatex to work
\usepackage{fontspec}
\usepackage{xunicode}

%% color for the links 
\usepackage[usenames,dvipsnames]{xcolor}
\definecolor{ImperialBlue}{rgb}{0.082,0.416,0.608}
\definecolor{ImperialLightBlue}{rgb}{0.545,0.682,0.8}

%% hyperlinks
\usepackage[xetex, 
	colorlinks=true,
	urlcolor=ImperialLightBlue,
	plainpages=false,
  	pdfpagelabels,
  	bookmarksnumbered,
  	pdftitle={\mytitle},
  	pagebackref,
  	pdfauthor={\myauthor},
  	pdfkeywords={\mykeywords}
  	]{hyperref}

%%%------------------------------------------------------------------------
%%% Document
%%%------------------------------------------------------------------------
\begin{document}

%% Choose fonts for use with xelatex
%% Minion and Myriad are widely available, from Adobe. 
%% Pragmata is available to buy at http://www.fsd.it/fonts/pragma.htm
%% and is worth every penny. Any good monospace font will work fine, though.
%% Consolas or inconsolata are good alternatives.
\setromanfont[Mapping={tex-text},Numbers={OldStyle},Ligatures={Common}]{Minion Pro} 
\setsansfont[Mapping=tex-text,Colour=156a9b]{Myriad Pro}
\setmonofont[Mapping=tex-text,Scale=0.9]{Consolas} % Pragmata, Lucida Console


%%%------------------------------------------------------------------------
%%% Local commands
%%%------------------------------------------------------------------------

%% Marginal header
%% Note: as the document goes on you may need to introduce a (gradually increasing)
%% \vspace element to keep the marginal header pleasingly aligned with the first 
%% item in the body text. Like this: \marginhead{{\vskip 0.4em}Grants}, or 
%% \marginhead{{\vskip 0.8em}Service}. Experiment as needed.
\newcommand{\marginhead}[1]{\marginpar{\textsf{{\footnotesize #1}}}}


%% custom ampersand (font consistent with the one chosen above)
\newcommand{\amper}{{\fontspec[Scale=.95,Colour=AA0000]{Minion Pro Medium}\selectfont\&\,}}

%% No bullets on labels
\renewcommand{\labelitemi}{~} 

%% Custom hanging indent for vita items
\def\ind{\hangindent=1 true cm\hangafter=1 \noindent}
%\def\ind{\hangindent=18pt\hangafter=1 \noindent}
\def\labelitemi{~}
\renewcommand{\labelitemii}{~}

%%%------------------------------------------------------------------------
%%% Page layout
%%%------------------------------------------------------------------------
\pagestyle{fancy}
\renewcommand{\headrulewidth}{0pt}
\fancyhead{}
\fancyfoot{}
\rhead{{\scriptsize\thepage}}

%% git revision control footer 
\rfoot{\textcolor{Gray}{\texttt{\scriptsize \VCRevision\ on \VCDateTEX}}} % git revision info inserted via external script -- see docs for vc package for details. comment out this line if you're not using vc, and also remove the \input{vc} line above.

%%%------------------------------------------------------------------------
%%% Address and contact block
%%%------------------------------------------------------------------------
\begin{minipage}[t]{2.95in}
\flushright {\footnotesize \href{http://www.imperial.ac.uk/civilengineering}{Department of Civil \& Env.\ Engineering} \\ Imperial College London \\ London, UK\\ \vspace{-0.04in} SW7 2AZ}   
\end{minipage}
\hfill     
%\begin{minipage}[t]{0.0in}
% dummy (needed here)
%\end{minipage}
\hfill
\begin{minipage}[t]{1.7in}
  \flushright \footnotesize Phone: \myphone \\ 
  Fax: \myfax  \\ 
  {\scriptsize  \texttt{\href{mailto:\myemail}{\myemail}}} \\
  {\scriptsize  \texttt{\href{\myweb}{\myweb}}}
\end{minipage}


\bigskip

\bigskip

%% Name 
\noindent{\Large {D\textsc{r} J\textsc{ames} K\textsc{eirstead}}} {\scriptsize \textsc{CEng MEI}}
\reversemarginpar
\raggedright

\bigskip

\bigskip

%% Appointments

\noindent\marginhead{Appointments}%
%
\emph{Imperial College London \vspace{0.01in}}

\ind 2011--Present. Lecturer, Dept.\ of Civil and Environmental Engineering.      

\ind 2009--2011. Research Fellow and Team Leader, BP Urban Energy Systems project.

\ind 2006--2009. Research Associate, BP Urban Energy Systems project. \vspace{0.02in}

\medskip

\emph{Institute of Industrial Science, University of Tokyo \vspace{0.01in}}

\ind October 2010.  Visiting Research Fellow.

\bigskip

%% Education

\noindent\marginhead{Education}%
%
\emph{University of Oxford \vspace{0.01in}}

\ind 2006.  DPhil, Energy Policy. \emph{Behavioural responses to photovoltaic systems in the UK domestic sector.}

\ind 2002.  MSc, Environmental Change and Management (Distinction).

\medskip

\noindent\emph{Queen's University, Kingston, Ontario, Canada\vspace{0.02in}}

\ind 2001. BSc Applied Science, Civil Engineering. (First Class Honours.) 

\bigskip
 
%% Publications
\noindent\marginhead{Publications}%
%
%\medskip
\noindent\emph{Books \vspace{0.01in}}

\ind James Keirstead and Nilay Shah (editors).  2013. \emph{Urban Energy Systems: an Integrated Approach}. London: Routledge/Earthscan.
%\normalsize

\bigskip

\emph{Journal articles \vspace{0.01in}} % 0.05
 
%% Use revnumerate environment if numbered publications are needed. 
%% (Include it above in the preamble).
%% \renewcommand{\labelenumi}{\textsc{a}\theenumi.}
%% \begin{revnumerate}
% Niels and me with GEA/UK data
% Stefan's review paper
% Buildings and Environment with Carlos
% DisP paper with Carlos
%
% Actually published here http://onlinelibrary.wiley.com/journal/10.1111/%28ISSN%291467-8667 ?
% \ind Aruna Sivakumar, James Keirstead, JW Polak. 2012. ``Integrated Modelling of the Demand \& Supply Vectors in Urban Energy Systems: Conceptual and Modelling Frameworks for the Development of a New Toolkit'', Computer-Aided Civil and Infrastructure Engineering.

% \ind Stefan Pfenninger, Adam Hawkes, and James Keirstead. In review. ``Energy systems modeling for twenty-..rst century energy policy'' submitted to \emph{Renewable and Sustainable Energy Reviews}.
%
% \ind Nouri Samsatli, James Keirstead, and Nilay Shah. In review. ``A Generic MILP Model for the Design of Urban Energy Systems'' submitted to \emph{Energy Conversion and Management}.
%
\ind James Keirstead and Mark Jennings. In review. ``Calculating carbon abatement costs for the global buildings sector with a regionally-disaggregated hybrid model'' submitted to \emph{Building and Environment}.  % Energy Economics if that fails?

\ind James Keirstead. 2013. ``\href{http://dx.doi.org/10.1016/j.enpol.2013.08.063}{Benchmarking urban energy efficiency in the UK}'' \emph{Energy Policy}.  In press.

\ind Clémence Morlet and James Keirstead. 2013. ``\href{http://dx.doi.org/10.1016/j.enpol.2013.06.085}{A comparative analysis of urban energy governance in four European cities}'' \emph{Energy Policy}, 61: 852--863.

\ind James Keirstead and Aruna Sivakumar. 2012. ``\href{http://dx.doi.org/10.1111/j.1530-9290.2012.00486.x}{Using activity-based modeling to simulate urban resource demands at high spatial and temporal resolutions}''. \emph{Journal of Industrial Ecology}, 16(6): 889--900.

\ind Paul Rutter and James Keirstead.  2012. ``\href{http://dx.doi.org/10.1016/j.enpol.2012.03.072}{A brief history and the possible future of urban energy systems}''. \emph{Energy Policy}, 50: 72--80.

\ind Carlos Calderon and James Keirstead. 2012. ``\href{http://dx.doi.org/10.1080/09613218.2012.680702}{Modelling frameworks for delivering low carbon cities: advocating a normalised practice}''. \emph{Building Research and Information}, 40(4): 504--517.

\ind James Keirstead and Carlos Calderon.  2012. ``\href{http://dx.doi.org/10.1016/j.enpol.2012.03.058}{Capturing spatial effects, technology interactions, and uncertainty in urban energy and carbon models: retrofitting Newcastle as a case-study}.''  \emph{Energy Policy}, 46: 253--267.

\ind James Keirstead, Mark Jennings, and Aruna Sivakumar. 2012. ``\href{http://dx.doi.org/10.1016/j.rser.2012.02.047}{A review of urban energy system models: approaches, challenges, and opportunities}.'' \emph{Renewable and Sustainable Energy Reviews}, 16(6): 3847--3866.

\ind James Keirstead, Nouri Samsatli, Nilay Shah and C\'{e}line Weber. 2012. ``\href{http://dx.doi.org/10.1016/j.energy.2011.06.011}{The impact of CHP (combined heat and power) planning restrictions on the efficiency of urban energy systems}.'' \emph{Energy}, 41(1): 93--103. 

\ind James Keirstead, Nouri Samsatli, A.\ Marco Pantaleo, and Nilay Shah. 2012. ``\href{http://dx.doi.org/10.1016/j.biombioe.2012.01.022}{Evaluating biomass energy strategies for a UK eco-town with an MILP optimization model}.'' \emph{Biomass and Bioenergy}, 39: 306--316.

\ind Hao Liang, Weiding Long, James Keirstead, Nouri Samsatli, and Nilay Shah. 2012. ``\href{http://dx.doi.org/10.4028/www.scientific.net/AMR.433-440.1338}{Urban Energy System Planning and Chinese Low-Carbon Eco-City Case Study}''. \emph{Advanced Materials Research}, 433--440: 1338--1345. 

\ind James Keirstead and Nilay Shah. 2011. ``\href{http://dx.doi.org/10.1016/j.compenvurbsys.2010.12.005}{Calculating minimum energy urban layouts with mathematical programming and Monte Carlo analysis techniques}.'' \emph{Computers, Environment and Urban Systems} 35(5):~368--377. 

\ind Hao Liang, James Keirstead, Nouri Samsatli, Nilay Shah, and WD Long. 2011. ``Application of a novel, optimisation-based toolkit (`SynCity') for urban energy system design in Shanghai Lingang New City.'' \emph{Energy Education Science and Technology Part A} 28(1):~311--318.

\ind James Keirstead. 2010. ``\href{http://dx.doi.org/10.1108/14777831011010829}{Applying service niche indicators to London's energy system}'' \emph{Management of Environmental Quality: An International Journal} 21(1):~6--19.

\ind James Keirstead and Niels Schulz. 2010. ``\href{http://dx.doi.org/10.1016/j.enpol.2009.07.025}{London and beyond: Taking a closer look at urban energy policy}'' \emph{Energy Policy} 38(9):~4870--4879.

\ind James Keirstead. 2009. ``\href{http://dx.doi.org/10.1016/j.eiar.2008.09.001}{Feeling lucky? Using search engines to assess perceptions of urban sustainability}'' \emph{Environmental Impact Assessment Review} 29(2):~87--95.

\ind James Keirstead and Matt Leach.  2008. ``\href{http://dx.doi.org/10.1002/sd.349}{Bridging the gaps between theory and practice: a service niche approach to urban sustainability indicators}'' \emph{Sustainable Development} 16(5):~329--340.

\ind James Keirstead. 2008. ``\href{http://dx.doi.org/10.1016/j.enpol.2008.09.019}{What changes, if any, would increased levels of low-carbon decentralised energy have on the built environment?}'' \emph{Energy Policy} 36(12):~4518--4521.

\ind James Keirstead. 2007. ``\href{http://dx.doi.org/10.1016/j.enpol.2007.02.019}{Behavioural responses to photovoltaic systems in the UK domestic sector}'' \emph{Energy Policy} 35(8):~4128--4141.

\ind James Keirstead. 2007. ``\href{http://dx.doi.org/10.1016/j.enpol.2006.08.003}{The UK domestic photovoltaics industry and the role of central government}'' \emph{Energy Policy} 35(4):~2268--2280.

\ind James Keirstead. 2006. ``\href{http://dx.doi.org/10.1016/j.enpol.2005.06.004}{Evaluating the applicability of integrated domestic energy frameworks in the UK}'' \emph{Energy Policy} 34(17):~3065--3077.

\bigskip

\noindent\emph{Book chapters \vspace{0.01in}}
% \renewcommand{\labelenumi}{\textsc{c}\theenumi.}
% \begin{revnumerate}

%\ind Various chapters in UES book?
\ind N Shah, L Vallejo, T Cockerill, A Gambhir, A Heyes, T Hills, M Jennings, O Jones, N Kalas, J Keirstead, C Khor, C Mazur, T Napp, A Strapasson, D Tong, J Woods. 2013. ``\href{http://www3.imperial.ac.uk/climatechange/publications/collaborative/halving-global-co2-by-2050}{Halving Global CO2 by 2050: Technologies and Costs}''.  Imperial College London.

\ind James Keirstead and Nilay Shah. 2013. ``The changing role of optimization in urban planning'' in \emph{Optimization, Simulation and Control}, edited by A.\ Chinchuluun, P.M.\ Pardalos, R.\ Enkhbat, S.\ Pistikopoulos.  Springer Series in Optimization and Its Applications.  Springer.

\ind James Keirstead and Nilay Shah. 2012. ``Urban Energy Systems Planning, Design and Implementation'' in \emph{Energizing Sustainable Cities: Assessing Urban Energy}, edited by A.\ Grubler and D.\ Fisk. London: Earthscan/Routledge.

\ind Arnulf Grubler, Xuemei Bai, Thomas Buettner, Shobhakar Dhakal, David Fisk, Toshiaki Ichinose, James Keirstead, Gerd Sammer, David Satterthwaite, Niels Schulz, Nilay Shah, Julia Steinberger, and Helga Weisz. 2012. ``Urban Energy Systems'' in \emph{The Global Energy Assessment}. Cambridge: Cambridge University Press.

\ind Stephen Hammer, James Keirstead, Shobhakar Dhakal, Jeanene Mitchell, Michelle Colley, Richard Gonzalez, Morgan Herve-Mignucci, Lily Parshall, Niels Schulz, Michael Hyams. 2011. ``Climate Change and Urban Energy Systems'' in \emph{Climate Change and Cities: First Assessment Report of the Urban Climate Change Research Network}, edited by C.\ Rosenzweig, W.D.\ Solecki, S.A.\ Hammer, S.\ Mehrotra.  Cambridge: Cambridge University Press, 85--112.

\ind Niels Schulz, Nilay Shah, David Fisk, James Keirstead, Nouri Samsatli, Aruna Sivakumar, C\'{e}line Weber, and Ellin Saunders. 2010. ``Mobilize: The SynCity Urban Energy Systems Model'' in \emph{Ecological Urbanism}, edited by M.\ Mostafavi and G.\ Doherty.  Baden, Lars M\"{u}ller Publishers.

\ind James Keirstead, Nouri Samsatli and Nilay Shah. 2010. ``SynCity: an integrated tool kit for urban energy systems modelling'' in \emph{Energy Efficient Cities: Assessment Tools and Benchmarking Practices}, edited by R.\ Bose.  Washington: World Bank, 21--42.

\ind James Keirstead. 2010. ``Total primary energy supply'' in \emph{Green Energy: An A-to-Z Guide}, edited by P.\ Robbins and D.\ Mulvaney.  Berkeley, CA: SAGE Publications, 426--428.

%\end{revnumerate}

\bigskip 

\noindent\emph{Conference papers \vspace{0.01in}}

% \ind Eduardo Abreu and James Keirstead. In review. ``Peak demand reduction from the grid in urban energy systems with renewable DG penetration: an analysis of Brazilian systems'' submitted to \emph{Energy for Sustainable Development}.

\ind James Keirstead. 2013. ``Introducing sustainable development to engineers with a simple mathematical model''.  Engineering Education for Sustainable Development conference.  Cambridge.

\ind N J D Graham, K Hellgardt, K S Wong, J Keirstead and Q Majali. 2013. ``Combining H$_2$ Generation and Grey Water Recycling at Household Scale'' 12th International Conference on Sustainable Energy Technologies, Hong Kong.

\ind Sara Giarola, A.\ Marco Pantaleo, James Keirstead, and Nilay Shah. 2013. ``Biomass and Natural Gas Cofiring Strategies for Optimal Integration into Urban Energy Systems''.  European Biomass Conference, Copenhagen.

\ind Carlos Calderon and James Keirstead. 2012. ``Modelling approaches for retrofitting energy systems: Newcastle as a case study''.  Applied Urban Modelling 2012, Cambridge.

\ind James Keirstead and Koen van Dam. 2011. ``A survey on the application of conceptualisations in energy systems modelling'' in \emph{Formal Ontologies Meet Industry: Proceedings of the Fifth International Workshop}, edited by P.E.\ Vermass and V.\ Dignum.  Amsterdam: IOS Press, 50--62.

\ind Salvador Acha, Koen van Dam, James Keirstead and Nilay Shah. 2011. ``Integrated modelling of agent-based electric vehicles into optimal power flow studies'' in \emph{Proceedings of the 21st International Conference and Exhibition on Electricity Distribution (CIRED 2011)}.  Frankfurt, 1--4.

\ind Koen van Dam and James Keirstead. 2010. ``Re-use of an ontology for urban energy systems modelling'' in \emph{Proceedings of Next Generation Infrastructures Eco-Cities conference}.  Shenzhen, 1--6.

\ind C\'{e}line Weber, James Keirstead, Nouri Samsatli, Nilay Shah and David Fisk. 2010. ``Trade-offs between Layout of Cities and Design of District Energy Systems'' in \emph{Proceedings of the 23rd International Conference on Efficiency, Cost, Optimization, Simulation and Environmental Impact of Energy Systems}.  Lausanne, 1--8.

\ind James Keirstead, Nouri Samsatli, Nilay Shah and C\'{e}line Weber. 2010. ``The implications of CHP planning restrictions on the efficiency of urban energy systems'' in \emph{Proceedings of the 23rd International Conference on Efficiency, Cost, Optimization, Simulation and Environmental Impact of Energy Systems}.  Lausanne, 1--8.

\ind Michael Kostantindis, Nouri Samsatli, James Keirstead and Nilay Shah. 2010. ``Modelling of integrated municipal solid waste to energy technologies in the urban environment'' in \emph{Proceedings of the 3rd International Conference on Engineering for Waste and Biomass Valorisation}.  Beijing, 1--6.

\ind James Keirstead and Koen van Dam. 2010. ``A comparison of two ontologies for agent-based modelling of energy systems'' in \emph{Proceedings of the First International Workshop on Agent Technologies for Energy Systems (ATES 2010)}, edited by A.\ Rogers. Toronto, 1--8. 

\ind James Keirstead, Nouri Samsatli and Nilay Shah. 2009. ``SynCity: an integrated tool kit for urban energy systems modelling'' in \emph{Proceedings of the 5th Urban Research Symposium}.  Marseilles, 1--19.

\ind James Keirstead, Nouri Samsatli, A.\ Marco Pantaleo, and Nilay Shah. 2009. ``Evaluating integrated urban biomass strategies for a UK eco-town'' in \emph{Proceedings of the European Biomass Conference}.  Hamburg, 2115--2127.

\ind James Keirstead. 2009. ``London's energy system: assessing the quality of urban sustainability indicators using the service niche approach'' in \emph{Proceedings of the Second International Conference on Whole Life Urban Sustainability and its Assessment}, edited by M.\ Horner, A.\ Price, J.\ Bebbington, R.\ Emmanuel.  Loughborough, 471--487.

\ind James Keirstead. 2009. ``London's energy system: prospects for using the service niche approach in current indicator practice'' in \emph{Proceedings of the Second International Conference on Whole Life Urban Sustainability and its Assessment}, edited by M.\ Horner, A.\ Price, J.\ Bebbington, R.\ Emmanuel.  Loughborough, 488--502.

\ind James Keirstead. 2007. ``Selecting sustainability indicators for urban energy systems'' in \emph{Proceedings of the International Conference on Whole Life Urban Sustainability and its Assessment}, edited by M.\ Horner, C.\ 
Hardcastle, A.\ Price, J.\ Bebbington. Glasgow, 1--20.

\ind James Keirstead. 2005. ``Household behavioural responses to photovoltaic-system monitoring devices'' in \emph{Proceedings of the World Renewable Energy Congress}, edited by M.\ Imbabi and C.\ Mitchell. Aberdeen, 458--463.

\ind James Keirstead. 2005. ``Photovoltaics in the UK domestic sector: a double-dividend?'' in \emph{Proceedings of the ECEEE Summer Study}, edited by F.\ Bartiaux and A.\ Saln{\ae}s. Mandelieu, France, 1249--1258.

\bigskip
 
%\newpage
\noindent\emph{Reviews, software and other writing \vspace{0.05in}}

%\renewcommand{\labelenumi}{\textsc{r}\theenumi.}
%\begin{revnumerate}

\ind James Keirstead. 2013. ``scholar: Analyse citation data from Google Scholar''. R package version 0.1.0, \url{http://cran.r-project.org/web/packages/scholar/}.

\ind James Keirstead. 2013. ``decctools: tools for accessing UK energy statistics''. R package version 0.1.1, \url{http://cran.r-project.org/web/packages/decctools/}.

\ind James Keirstead. 2010. ``Identifying lessons for energy-efficient cities using an integrated urban energy systems model''.  Working paper prepared for \emph{The Global Energy Assessment}.  Laxenburg, Austria: IIASA, 1--23.

\ind James Keirstead. 2010. ``\href{http://jasss.soc.surrey.ac.uk/13/1/reviews/keirstead.html}{Review} of Ball, Michael and Wietschel, Martin (eds): The Hydrogen Economy: Opportunities and Challenges'' \emph{Journal of Artificial Societies and Social Simulation}, 13(1).

\ind James Keirstead. 2008--2010.  Contributor to \href{http://www.academicproductivity.com}{www.academicproductivity.com}

\ind James Keirstead. 2008. ``\href{http://www.stockholm-network.org/downloads/publications/Climate_of_Opinion_10.pdf}{Victims or leaders? Cities and global climate policy}''. \emph{Climate of Opinion}, 10:~4--5.
% Also mention participation in Stockholm Network's Carbon Scenarios project?

\ind James Keirstead. 2007.  \href{http://www3.imperial.ac.uk/pls/portallive/docs/1/24897696.PDF}{\emph{Towards urban energy system indicators}}.  Internal report. London: Imperial College.

\ind James Keirstead. 2005. ``Energy matters: Power from the sun''.  \emph{Geography Review}, September 2005: 17--19.

\ind James Keirstead. 2004. ``Negotiating lifestyles: understanding lifestyles can help achieve sustainability''. \emph{IISD Alumni Inside}.

% %\end{revnumerate}
 \bigskip

%% Presentations
\noindent\marginhead{Invited Talks}%
%
% TedX London, July 2013
\ind 2013. ``How do models of urban energy systems account for climate change?''. RCN Virtual Collaboratory Call, October.

\ind 2013. ``Technologies and policies for urban energy systems''.  Global Sustainability Summer School, Potsdam Institute for Climate Impact Research, Potsdam, Germany. July.

\ind 2013. ``Fit for purpose?  A comparison of urban resource demand simulation techniques''.  ISIE Conference, Ulsan, South Korea.

\ind 2013. ``What can dynamical systems tell us about urban energy systems?''.  SIAM Conference on the Application of Dynamical Systems, Snowbird, Utah. 

\ind 2012. ``What next for urban energy systems?'' Flexible Energy Delivery Systems seminar series, Cardiff University. October.

\ind 2012. ``Modelling urban energy systems with SynCity'' RCN Workshop on Sustainable Cities, Denver, USA.  August.

\ind 2012. ``Modelling urban energy systems'' GreenBridge seminar series, University of Cambridge, UK. March.

\ind 2012. ``Modelling approaches to urban energy systems: optimization, simulation, and more'' Martin Centre seminar series, University of Cambridge, UK. February.

\ind 2011. ``Modelling urban energy systems'' National Academy of Engineering/EU Frontiers of Engineering symposium, Irvine, CA. November.

\ind 2011. ``Modelling urban energy transitions'' Grantham Institute, Imperial College, London. September.

\ind 2011. ``Approaches, challenges and opportunities in urban energy systems modelling'' IIASA, Laxenburg, Austria. March.

\ind 2010. ``Challenges in multi-scale urban energy modelling and data collection: What's `just right' for urban energy data?'' \textsc{iis} seminar, University of Tokyo. November.

\ind 2010. ``Modelling energy systems with ontologies'' \textsc{iam} seminar, University of Southampton. June.

\ind 2010. ``SynCity: New Tools, Methods and Business Models for future urban sustainability'' Communities for Advanced Distributed Energy Resources, San Diego.  April.

\ind 2010. ``Lessons from SynCity: insights on the modelling of urban energy systems'' Gordon Conference on Industrial Ecology, New London, NH.  July.

\ind 2009. ``The SynCity layout model: exploring the limits of low-carbon urban design'' University of Tokyo, Tokyo.  September.

\ind 2009. ``Modelling urban energy systems with SynCity'' Nagoya University, Nagoya. February.

\ind 2008. ``SynCity: an integrated framework for modelling urban energy systems'' Asian Institute of Technology, Bangkok. February.

\ind 2007. ``Urban Energy Transitions'' National Institute for Environmental Studies, Tsukuba.  March.

%\end{revnumerate}

\bigskip

\noindent\marginhead{Grants and \newline Awards}%
%\medskip
%
% FLIRE?
%
\ind 2012--present.  SusLabs NWE, EU Interregio.  Co-investigator (£170,000).

\ind 2012. Smart Thermal Storage Systems, DECC.  Co-investigator (£30,000).

\ind 2012. The cost of 2$^\circ$C, AREVA.  Co-investigator.

\ind 2012. Energy Futures Lab Research Challenge, Imperial College.  Principal investigator (£8,000).

% Commented out as I didn't do that much on it
%\ind 2011. Biomass system value-chain modelling, Energy Technologies Institute.  Co-investigator. (£153,000).
\ind 2011. Energy systems analysis of South Heaton, Newcastle University.  Co-investigator (£2,500).

\ind 2009. MBA Innovation, Entrepreneurship and Design, Imperial College.  Ideator (30 person-months research time).

\ind 2003--2006.  Commonwealth Scholar, held at University of Oxford.

\ind 2001--2002.  British Council Chevening Scholar, held at University of Oxford.

\ind 2001.  Consulting Engineers of Ontario Louanne Smrke Award for excellence in written communication.

\ind 2001. CW Marshall Award for excellence in structural engineering, Queen's University.

\ind 2000.  Guinness Book of World Records, longest distance travelled in a solar vehicle (7044 km).

\ind 2000. One of 125 ``Canadians Shaping the Nation We Will Live in Tomorrow'', chosen by \emph{The Globe and Mail}.

\ind 2000.  Fifth Field Company Prize for excellence in hydrology, Queen's University.

\ind 2000. Frederick and Christopher Ansley Award for extra-curriculars and academic excellence, Queen's University.

\ind 1999.  \textsc{sn} Graham Award for extra-curriculars and academic excellence, Queen's University.

\ind 1998--2001.  Dean's Scholar for academic excellence, Queen's University.

\bigskip 

\noindent\marginhead{Service to the \newline Profession}%
% \ind 2013 
\ind 2013.  Invited participant.  Research Councils UK energy strategy workshop on energy infrastructure.  Birmingham.

\ind 2013.  Technical committee member.  International Society for Industrial Ecology conference.

\ind 2011--present. Board member, Sustainable Urban Systems section, International Society for Industrial Ecology.

\ind 2011--2012. Technical advisory panel member, Challenging Lock-in Through Urban Energy Systems EPRSC project.

\ind 2011--2012. Technical committee member, CIBSE ASHRAE Technical Symposium 2012.

\ind 2011--2013. Programme committee member, 2nd--4th International Workshops on Agent Technologies for Energy Systems (ATES).

\ind 2010 Reviewer, ECOS and Next Generation Infrastructures conferences.

\ind 2009. Invited participant.  EPSRC Grand Challenge workshop on Sustainable Urban Environments.

\ind Reviewer, Engineering and Physical Sciences Research Council, Economic and Social Research Council, US National Science Foundation, Natural Sciences and Engineering Research Council of Canada, Social Sciences and Humanities Research Council of Canada.

\ind Reviewer, \emph{Energy Policy}, \emph{Proceedings of the National Academy of Sciences}, \emph{Building Research and Information}, \emph{Technological Forecasting and Social Change}, \emph{Energy}, \emph{Energy Economics}, \emph{Environmental Practice}, \emph{Journal of Artificial Societies and Social Simulation}, \emph{Energy Efficiency}, \emph{Sustainable: Science, Practice and Policy}, \emph{The Energy Journal}, \emph{Journal of Industrial Ecology}, \emph{Computers, Environment and Urban Systems}, \emph{Journal of Urban Technology}, \emph{International Journal of Sustainable Transportation}, \emph{Climatic Change}, \emph{Urban Design and Planning}, \emph{Environmental Science \& Technology}, \emph{IET Renewable Power Generation}, \emph{Environment and Planning B}, \emph{Ecological Modelling}, \emph{Environmental Research Letters}, \emph{Journal of Environmental Planning and Management}, \emph{Cities}, \emph{Sustainability Science}, \emph{Urban Studies}.

\ind Consultant, Environmental Change Institute, University of Oxford (Carbon Trust, Department for International Development), Foresight, International Institute for Applied Systems Analysis, EDF/Weber Shandwick, The Economist Group, Energy Technologies Institute, Arup.
\bigskip%

\newpage
\noindent\marginhead{Teaching}%
%
\ind 2013--present.  CI{\addfontfeatures{Numbers={Lining}}1-101} Drawing

\ind 2011--present.  CI{\addfontfeatures{Numbers={Lining}}1-182} Energy Systems

\ind 2011--present.  CI{\addfontfeatures{Numbers={Lining}}9-SD} Sustainable Development MSc module

\ind 2007--present. Various short lectures and modules on urban energy systems at both undergraduate and graduate levels.
 
\ind 2007--present. Supervised or co-supervised 29 MSc projects, 5 MEng projects and internships.

\bigskip
\noindent\marginhead{PhD students}%
%%
\ind 2012--present.  Stefan Pfenninger. \emph{Multi-scale energy systems}

\ind 2012--present.  Tom Ravalde. \emph{Highly integrated urban resource systems}

\ind 2012--present.  Rembrandt Koppelaar. \emph{Expert knowledge development of property allocation and technological change} 

\ind 2012--present.  Pantelis Broukos. \emph{Multi-period urban energy systems optimization in the context of water and wastewater management in Luxembourg}
\bigskip

%\newpage
\noindent\marginhead{Affiliations}%
%
\ind 2010--present.  Member and Chartered Energy Engineer, Energy Institute.

\ind 2010--present.  Member, International Society for Industrial Ecology.

\ind 2009--present.  Associate, Higher Education Academy.

\ind 2008--present.  Member, British and International Institutes for Energy Economics.

\end{document}
